Now we have the tools to tackle the main theorem. 
\section{Main theorem}

Our goal is to get a snake lemma type connecting homomorphism for bigger diagrams than the standard $2\times3$. The idea of the proof is to take a bicomplex which is exact almost everywhere. In this way, when we construct its associated spectral sequence, we mostly get 0's everywhere. We leave room for the bicomplex to be non-exact at places such that when we ``flip'' to the $n$'th and final page, we have a single morphism which is then forced to be an isomorphism. Then we use that isomorphism to construct the long connecting homomorphism. 

\begin{theorem}[The anaconda lemma]
\label{Thm:long}
Let $n \geq 2$ and
\begin{center}
    \begin{tikzcd}
& {C_{n+1,n}} \arrow[r]{}{d_{n+1,n}^h} \arrow[d]{}{d^v_{n+1,n}} & {C_{n,n}} \arrow[r]{}{d^h_{n,n}} \arrow[d]{}{d^v_{n,n}} & \cdots \arrow[r] & {C_{1,n}} \arrow[d]{}{d^v_{1,n}} \arrow[r] & 0 \\
0 \arrow[r] & {C_{n+1,n-1}} \arrow[r]{}{d^h_{n+1,n-1}} \arrow[d] & {C_{n,n-1}} \arrow[r] \arrow[d] & \cdots \arrow[r] & {C_{1,n-1}} \arrow[d] \arrow[r] & 0 \\
\vdots            & \vdots \arrow[d]              & \vdots \arrow[d]              &                  & \vdots \arrow[d]    \\
0 \arrow[r] & {C_{n+1,2}} \arrow[r]{}{d^h_{n+1,2}} \arrow[d]{}{d^v_{n+1,2}} & {C_{n,2}} \arrow[r] \arrow[d]{}{d^v_{n,2}} & \cdots \arrow[r] & {C_{1,2}} \arrow[d]{}{d^v_{1,2}} \arrow[r] & 0       \\
0 \arrow[r] &{C_{n+1,1}} \arrow[r]           & {C_{n,1}} \arrow[r]           & \cdots \arrow[r]{}{d_{2,1}^h} & {C_{1,1}}   
\end{tikzcd}
\end{center}
be a bicomplex with $n$ exact rows and $n+1$ exact columns. Then we have an exact sequence
\begin{center}
\begin{tikzcd}[column sep=small]
0 \ar[r] & \Ker(d_{n+1,n}^h)\ar[r] & \Ker(d^v_{n+1,n})\ar[r] & \cdots \ar[r] & \Ker(d^v_{1,n}) \arrow[dllll, overlay, out=350,in=170, looseness=1]{}{}\\
\Cok(d^v_{n+1, 2}) \ar[r] & \cdots \ar[r] & \Cok(d^v_{1,2})\ar[r] & \Cok(d^h_{2,1})\ar[r] & 0
\end{tikzcd}
\end{center}
\end{theorem}

\begin{proof}
%By the proof of the normal snake lemma that we just did, we see that the connecting homomorphism in fact comes from a differential on the second page of the spectral sequence. So instead of just having a differential on the $E_2$ page, we extend this to a differential on some $E_n$ page.
By inserting a kernel, a cokernel and two zeroes, we can make the bicomplex above into the following bicomplex. 
\begin{center}
\begin{tikzcd}[column sep=1em]
\Ker(d_{n+1,n}^h) \arrow[r] \arrow[d] & {C_{n+1,n}} \arrow[r] \arrow[d] & {C_{n,n}} \arrow[r] \arrow[d] & \cdots \arrow[r] & {C_{1,n}} \arrow[r] \arrow[d] & 0   \arrow[d] \\
0 \arrow[r] \arrow[d] & {C_{n+1,n-1}} \arrow[r] \arrow[d] & {C_{n,n-1}} \arrow[r] \arrow[d] & \cdots \arrow[r] & {C_{1,n-1}} \arrow[d] \arrow[r] & 0  \arrow[d]  \\
\vdots \arrow[d] & \vdots \arrow[d]              & \vdots \arrow[d]              &                  & \vdots \arrow[d]              &  \vdots \arrow[d] \\
0 \arrow[r]          & {C_{n+1,1}} \arrow[r]           & {C_{n,1}} \arrow[r]           & \cdots \arrow[r] & {C_{1,1}} \arrow[r]           & \Cok(d_{2,1}^h)
\end{tikzcd}
\end{center}
From here we see that all rows are exact, so applying homology in the horizontal direction first in our spectral sequence gives us that the $E^{\infty}$-page should be all zeroes. Starting with vertical homology we get the $E^1$-page to be
\begin{center}
\begin{tikzcd}[column sep=1em]
\Ker(d_{n+1,n}^h) \arrow[r]{}{i} & \Ker(d^v_{n+1,n}) \arrow[r]{}{g_{n+1}} & \Ker(d^v_{n,n}) \arrow[r]{}{g_n} & \cdots \arrow[r]{}{g_2} & \Ker(d^v_{1,n}) \ar[r] & 0 \\
0&0                  & 0                  & \cdots           & 0                  & 0        \\
\vdots&\vdots             & \vdots             &                  & \vdots             &  \vdots  \\
0&0                  & 0                  & \cdots           & 0                  & 0        \\
0 \ar[r] &\Cok(d^v_{n+1, 2}) \arrow[r]{}{h_{n+1}} & \Cok(d^v_{n,2}) \arrow[r]{}{h_n} & \cdots \arrow[r]{}{h_2} & \Cok(d^v_{1, 2}) \arrow[r]{}{j} & \Cok(d_{2,1}^h)
\end{tikzcd}
\end{center}
And then the $E_2$ page to be:
\begin{center}
\begin{tikzcd}[column sep=1em]
\Ker(i)&\displaystyle\frac{\Ker(g_{n+1})}{\Image(i)}            & \displaystyle\frac{\Ker(g_{n})}{\Image(g_{n+1})} & \cdots & \displaystyle\frac{\Ker(g_2)}{\Image(g_{3})} & \Cok(g_2) \\
&0                         & 0                          & \cdots & 0                          & 0        \\
&\vdots                    & \vdots                     &        & \vdots                     &          \\
&0                         & 0                          & \cdots & 0                          & 0        \\
&\Ker(h_{n+1})  & \displaystyle\frac{\Ker(h_{n+1})}{\Image(h_n)} & \cdots & \displaystyle\frac{\Ker(h_2)}{\Image(h_{3})} & \displaystyle\frac{\Ker(j)}{\Image(h_{2})} & \Cok(j)
\end{tikzcd}
\end{center}
Since the differentials on page $r$ is a morphism from $E_{p,q}^r$ to $E_{p+r,q+r-1}^r$, there can be no non-zero differentials on the following pages, except on the $E_n$-page. This is because we see that $E_{p,q}^r=0$ for all $q\in \{1,\dots,n\}, p\in \{2,\dots, n-1\}$. This means that all of the differentials either pass out from, or into a 0. Hence all the differentials are zero, and the second page is equal to the third, and the fourth, and so on, until we get to the $n$-th page. Here, finally we have a differential long enough to reach from the top row to the bottom one. This differential, $\Psi : \Ker(h_{n+1})\longrightarrow \Cok(g_2)$ is the longest differential we are going to get. 
\begin{center}
\begin{tikzcd}[column sep=1em]
\Ker(i)&\displaystyle\frac{\Ker(g_{n+1})}{\Image(i)}            & \displaystyle\frac{\Ker(g_{n})}{\Image(g_{n+1})} & \cdots & \displaystyle\frac{\Ker(g_2)}{\Image(g_{3})} & \Cok(g_2) \\
&0                         & 0                          & \cdots & 0                          & 0        \\
&\vdots                    & \vdots                     &        & \vdots                     &          \\
&0                         & 0                          & \cdots & 0                          & 0        \\
&\Ker(h_{n+1}) \arrow[rrrruuuu]{}{\Psi} & \displaystyle\frac{\Ker(h_{n+1})}{\Image(h_n)} & \cdots & \displaystyle\frac{\Ker(h_2)}{\Image(h_{3})} & \displaystyle\frac{\Ker(j)}{\Image(h_{2})} & \Cok(j)
\end{tikzcd}
\end{center}
Since our spectral sequence is a first quadrant spectral sequence from a bicomplex, we know that it converges to the homology of the totalization of the bicomplex with respect to both of the natural filtrations. We showed above that the spectral sequence converges to zero when we start with the horizontal homology, so then it has to converge to zero when starting with the vertical homology as well. This means that we have $E_{*,*}^{n+1} = E_{*,*}^{\infty} = 0$. For this to be the case we must have that everything on the $E_{*,*}^n$-page, except $\Ker(h_{n+1})$ and $\Cok(g_2)$ to be zero, and that the differential $\Psi$ must be an isomorphism. Since all the homology groups in the top and bottom row are zero, this means that we have exact sequences 

\begin{center}
\begin{tikzcd}[column sep=1.8em]
0 \arrow[r] & \Ker(d_{n+1,n}^h) \arrow[r]{}{i} & \Ker(d^v_{n+1,n}) \arrow[r]{}{g_{n+1}} & \cdots \arrow[r]{}{g_3}& \Ker(d^v_{2,n}) \arrow[r]{}{g_2} & \Ker(d^v_{1,n}) \end{tikzcd}
\end{center}
and
\begin{center}
\begin{tikzcd}[column sep=1.2em]
\Cok(d^v_{n+1,2}) \arrow[r]{}{h_{n+1}} & \Cok(d^v_{n,2}) \arrow[r]{}{h_n} & \cdots \arrow[r]{}{h_2} & \Cok(d^v_{1,2}) \arrow[r]{}{j} & \Cok(d_{2,1}^h) \ar[r] & 0 \end{tikzcd}
\end{center}
The connecting homomorphism comes from inverting our differential $\Psi$. We then have an exact sequence
\begin{center}
\begin{tikzcd}[column sep=1.2em]
0 \arrow[r]    & \Ker(d^h_{n+1,n}) \arrow[r] & \cdots \arrow[r]                               & \Ker(d^v_{2,n}) \arrow[r] & \Ker(d^v_{1,n}) \arrow[dllll, overlay, out=350,in=170, looseness=1, dashed]{}{\partial} \\

\Cok(d^v_{n+1,2}) \arrow[r] & \Cok(d^v_{n,2}) \arrow[r] & \cdots \arrow[r]                               & \Cok(d_{2,1}^h) \arrow[r] & 0                                     
\end{tikzcd}
\end{center}
If we let $\pi$ be the epimorphism $\pi: \Ker(d^v_{1,n})\to \Cok(g_2)$ and $\iota$ be the monomorphism $\iota: \Ker(h_{n+1})\to \Cok(d^v_{n+1,2})$, then we have $\partial = \iota \circ \psi^{-1}\circ \pi$. We have then joined together our two exact sequences into one longer exact sequence, which is the sequence we wanted. 
\end{proof}

%               &                & \Cok(h_{n+1}) \cong \Ker(g_2) \arrow[lld, hook'] &                &                                        \\
