\begin{corollary}[Snake lemma]
The normal snake lemma follows from setting $n=2$. We do remark that this proof of the snake lemma is circular, as the snake lemma is needed to prove the convergence of the spectral sequence. 
\end{corollary}

\begin{corollary}
Every surjection-injection composition can be realised as a connecting homomorphism in a diagram. 
\end{corollary}
\begin{proof}
This is inspired by the fact that we can realise each morphism $f$ between two objects $A$ and $B$ in an abelian category $\mathscr{A}$ as the connecting homomorphism in the following diagram.
% https://tikzcd.yichuanshen.de/#N4Igdg9gJgpgziAXAbVABwnAlgFyxMJZABgBpiBdUkANwEMAbAVxiRGJAF9T1Nd9CKAIzkqtRizYBBLjxAZseAkQBMo6vWatEIGd16KBRMkLGbJOgEKyD-ZcNKmNE7SGv75fJYORqn4rTYOTjEYKABzeCJQADMAJwgAWyQyEBwIJCEPeKTM6nSkNQCLECwoAH09ORzkxFSCxABmbITakTSMxAAWZ0CdGJsQGsL8zoBWFtym0aQe4tcy8vdq1tmZxAmKTiA
\begin{equation*}
\begin{tikzcd}
0 \arrow[r] \arrow[d] & A \arrow[r, "id_A"] \arrow[d, "f"] & A \arrow[d] \\
B \arrow[r, "id_B"]   & B \arrow[r]                        & 0          
\end{tikzcd}
\end{equation*}
We can then take a composition of morphisms
\begin{tikzcd}
A \arrow[r, "f"] & B \arrow[r, "g"] & C
\end{tikzcd}
and construct a diagram realising $g\circ f$ as the connecting homomorphism by using these as building blocks. For the morphisms $f$ and $g$ separately, we get diagrams 
\begin{center}
    \begin{tikzcd}
0 \arrow[r] \arrow[d] & A \arrow[r, "id_A"] \arrow[d, "f"] & A \arrow[d] \\
B \arrow[r, "id_B"]   & B \arrow[r]                        & 0          
\end{tikzcd}
and 
\begin{tikzcd}
0 \arrow[r] \arrow[d] & B \arrow[r, "id_B"] \arrow[d, "g"] & B \arrow[d] \\
C \arrow[r, "id_C"]   & C \arrow[r]                        & 0          
\end{tikzcd}
\end{center}
which we can stack on top of each other to form the following staircase diagram.
\begin{equation*}
    \begin{tikzcd}
                      & 0 \arrow[r] \arrow[d]              & A \arrow[r, "id_A"] \arrow[d, "f"] & A \arrow[d] \\
0 \arrow[r] \arrow[d] & B \arrow[r, "id_B"] \arrow[d, "g"] & B \arrow[d] \arrow[r]              & 0           \\
C \arrow[r, "id_C"]   & C \arrow[r]                        & 0                                  &            
\end{tikzcd}
\end{equation*}
Since $f$ is surjective and $g$ is injective, we get that the diagram is an exact bicomplex, and by \cref{Thm:long} we get a homomorphism $\partial :A \longrightarrow C$, which by going down the staircase in the diagram is the same as the composition $g\circ f$. 


\end{proof}

%Burde være, men nå kommuterer ikke firkantene lenger :(:
%\begin{equation*}
%\begin{tikzcd}
%\Ker g \ar[r] \ar[d]  & \Ker g \arrow[r, "0"] \arrow[d]              & A \arrow[r, "id_A"] \arrow[d, "f"] & A \arrow[d] \\
%0 \arrow[r] \arrow[d] & B \arrow[r, "id_B"] \arrow[d, "g"] & B \arrow[d] \arrow[r]              & 0  \ar[d]         \\
%C \arrow[r, "id_C"]   & C \arrow[r, "0"]                        & \Cok f \ar[r]                                  &     \Cok f       
%\end{tikzcd}
%\end{equation*}