\section{Introduction}

In an introductory course on homological algebra one learns about the snake lemma. It's famous for producing results in homological algebra, most notably for producing connecting homomorphisms. The lemma first showed up in D. A. Buchsbaums article ``Exact Categories and Duality'' from 1955 as lemma 5.8. More and more refined proofs have later arrived, and the standard method for proving it in your typical homological algebra class is to use diagram chasing. Using actual elements can feel cheap sometimes, and invoking the powerful Freyd-Michell's embedding theorem can feel like cheating to get to the point where we can use elements. The snake lemma is intuitive, but for students the connecting homomorphism feels a bit like a magic homomorphism that just happens to show up. 

There is an object that is almost entirely built out of connecting homomorphisms, and this is the spectral sequence. The study of spectral sequences arises very naturally when doing algebraic topology and it is one of the standard techniques for computing the homology of extensions, how products of derived functors work and most important for this article, computing the homology of the total complex of a bicomplex. Luckily for us, spectral sequences do not require the use of elements, but in the end of the article, when explicitly constructing the long connecting homomorphism, we will be invoking the Freyd-Michell's embedding theorem.